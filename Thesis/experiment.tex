\chapter{Experiment and Result}
This chapter shows the experiment for Chinese relational pair extraction.
First, the data source and data generating process will be explained.
Then the experimentl setting will be presented in the second section.
In the last 2 sections, the result for testing the proposed framework with given data will be discussed.

\section{Dataset}
Distant supervision learning, which is an semi-supervised framework, uses small size of labelled data and huge size of unlabelled data for training.
In relation extraction work, the labelled data are usually from an existing dataset and the unlabelled data are from a huge corpus containing many sentences, such as news corpus or Wikipedia.
In this thesis, we use Chinese entities pairs from ConceptNet as labeled data and sentences in Taiwan Sinica Corpus as unlabeled data.

\subsection{ConceptNet}
ConceptNet is developed by MIT Media Lab\cite{ConceptNet5}, containing only English data at beginning.
It is a graphical structure, with node representing $concept$, and link representing $relation$, describing a commonsense knowledge base.
A commonsense fact, formulated as two concepts and one relation bwtween them, is called $assertion$, which is stored as 2 nodes ($concepts$) and 1 link ($relation$) in the ConceptNet graph.
The knowledge in ConceptNet is originally provided by human as crowesourcing.
Afterwards, the authors imported knowledge from other datasets such as WordNet, Wikipedia, Wikitionary, DBPedia, and other sources.
From ConceptNet 3, by collaborating with universities from different countries, ConceptNet absorbed data from other languages, becoming a multilingual commonsense dataset.
Chinese assertions are collected from an online pet game, by feeding the pets new knowledge.
\par
Before ConceptNet 4, the relations of ConceptNet is predefined.
When collecting knowledge in Chinese, only 15 relations are considered (shown in Table \ref{tab:relation_list}).
The relation $AtLocation$ represents the relationship between an object and a location. For example, $AtLocation(Taipei, Taiwen)$ indicates that ``Taipei is located in Taiwan''.
\par
Each assertion is regarded as labelled data when training a distant supervision model.
The experiment uses Chinese assertions related to the relation $AtLocation$.
Since Chinese assertions in ConceptNet are generated from online users, the reliability is not guaranteed.
Among all 32816 assertions, only about $37\%$ of data are valid for training.
The examples of seeds from ConceptNet are shown in Table \ref{tab:AtLocation_seed}.

\begin{table}[t]
\begin{center}
\begin{tabular}{lcc}
\hline
Relation Name   & Number of Assertions\\  
\hline
AtLocation  & 32816\\
CausesDesire & 19408\\
HasProperty & 6822\\
NotDesires & 23930\\
UsedFor & 13548\\
Causes & 77336\\
HasSubevent & 40655\\
PartOf & 6159\\
Desires & 21772\\
IsA & 16094\\
HasFirstSubevent & 12046\\
MadeOf & 16357\\
CapableOf & 27444\\
SymbolOf & 4736\\
MotivatedByGoal &56636\\
\hline
\end{tabular}
\caption[List of relations corresponding to Chinese knowledge in ConceptNet]{\small List of relations corresponding to Chinese knowledge in ConceptNet}
\label{tab:relation_list}
\end{center}
\end{table}

\begin{table}[t]
\begin{center}
\begin{tabular}{| c | l | l |}
\hline
label  &   entity (object)   &    entity (location)\\
\hline
O   &   教授(professor)   &   研究所 (graduate school)\\
O   &   員工 (staff)   &   公司 (company)\\
O   &   人民 (people)   &   台灣 (Taiwan)\\
O   &   學生 (student)   &   學校 (school)\\
O   &   病人 (patient)   &   醫院 (hospital)\\
O   &   鯉魚 (carp)   &   池塘 (pond)\\
X   &   美國 (USA)   &   亞洲 (Asia)\\
X   &   程式 (program)   &   電腦 (computer)\\
X   &   太陽 (sun)   &   夏天 (summer)\\
X   &   觀眾 (audience)   &   電視 (television)\\
\hline
\end{tabular}
\caption[Example of positive and negative entity pairs of Chineses knowledge in ConceptNet]{\small Example of positive and negative entity pairs of Chineses knowledge in ConceptNet}
\label{tab:AtLocation_seed}
\end{center}
\end{table}

%==========我是ConceptNet的分隔線==========%
\subsection{Sinica Corpus}
The full name of Sinica Corpus\cite{Sinica_corpus} is ``Academia Sinica Balanced Corpus of Modern Chinese'', developed from 1994.
This corpus contains articles from 1981 to 2007, spread in different fields including philosophy, science, society, art, life, and literature.
The source of articles is also diverse, including news, book, textbook, magazine, and so on, which shows the corpus contains different styles of articles.
Each article is stored as sentences by segmenting the article with comma, period, semicolon, question mark and exclamation mark.
In Chinese articles, the usage of comma is slightly different from English.
Sometimes, the sentence before the comma and after the comma can be independent in grammar while they may be dependent in meaning.
The corpus is separated as about 6 hundred thousand sentences and each sentence is segmented as words.
Each word is annotated with a part-of-speech tag.
Since the articles of Sinica Corpus are cross many years, the phrasing may be slightly diferent to the wording nowadays.
The wording is more conscientious and careful because the sources are formal media.
The example of sentences are shown on Table \ref{tab:sa_example}
\par
The experiment uses the sentences in Sinica Corpus as unlabelled data.
Each sentences are regarded as a list of words and each 2 words in the sentence may convey one or no relation.
The method for generating training data from sentences is described in Section \ref{sec:method_label} on page \pageref{sec:method_label}.

\begin{table}[t]
\begin{center}
\begin{tabular}{| m{10cm} | m{6cm} |}
\hline
Sentence   &   Translation\\
\hline
{\small 民族學(Na) 研究所(Nc) 應(D) 主持(VC) 之(DE) 「(PARENTHESISCATEGORY) 台灣(Nc) 與(Caa) 東南亞(Nc) 土著(Na) 文化(Na) 與(Caa) 血緣(Na) 關係(Na) 」(PARENTHESISCATEGORY) 主題(Na) 研究(Nv) 計劃(Na) 之(DE) 需要(Na) ,(COMMACATEGORY)}  &   {According to the need for hosting the research project ``aboriginal culture and blood relationship bwtween Taiwan and Southeast Asia'', Institute of Ethnology ...}\\
\hline
{\small 邀請(VC) 蘇聯(Nc) 國家(Na) 科學院(Nc) 世界(Nc) 文學(Na) 研究所(Nc) 研究員(Na) Boris(FW) Parnickel(FW) 教授(Na) 於(P) 六月(Nd) 十九日(Nd) 至(P) 廿六日(Nd) 來訪(VA) ,(COMMACATEGORY)}  &   {... invited the researcher, Professor Boris Parnickel, who comes from Institute of World Literature of Russian Academy of Sciences, to visit from 19th to 26th June.}\\
\hline
{\small Boris(FW) 教授(Na) 之(DE) 專長(Na) 為(VG) 東南(Ncd) ((PARENTHESISCATEGORY) 特別(VH) 是(SHI) 馬來亞(Nc) 及(Caa) 印尼(Nc) )(PARENTHESISCATEGORY) 的(DE) 神話(Na) 傳說(Na) 及(Caa) 民俗(Na) ,(COMMACATEGORY)}   &   {Professor Boris specializes in the legend and folklore in Southeast Asia, especially in Malaysia and Indonesia, ...}\\
\hline
{\small 在(P) 此(Nep) 一(Neu) 領域(Na) 已(D) 有(V\_2) 傑出(VH) 之(DE) 研究(Nv) 成果(Na) 。(PERIODCATEGORY)}   &   {... having excellent achievements in this area.}\\
\hline
\end{tabular}
\caption[Example of sentences in Sinica Corpus]{Example of sentences in Sinica Corpus}
\label{tab:sa_example}
\end{center}
\end{table}

%==========我是平衡語料庫分隔線==========%
\subsection{UDN News Dataset}
\par
UDN.com is a news website in Taiwan, reporting various types of online news.
Different from Sinica Corpus, articles in UDN news are less formal, including many novel terms.
Sentences in UDN news are used as unlabelled data.
Articles are segmented to sentences by a period stop, which may conserve complete meaning in the sentence.
The articles are raw texts without any label or processing.
Before training, these sentences have to be segmented to words, labelled with part-of-speech tag, and other grammatically parsed with external NLP tools.
\par
There are two sets of UDN data collected for experiment.
One is recent news on the real time news column (from February to May 2015), and another is historical news in the past 10 years, including 3 different columns.
The difference of these 2 dataset is the size and age of news.

\section{Experiment setting}

\subsection{Experiment 1: Feature selection}
\subsection{Experiment 2: MIL Algorithm selection}
\subsection{Experiment 3: Iteration}
\subsection{Experiment 4: Multiple Relations}

\section{Evaluation}
We use $Accuracy$ for evaluating the result of cross validation:

\begin{equation}
Accuracy = \frac{\abs{accuate\ testing\ data}}{\abs{total\ testing\ data}}
\label{eq:accuracy}
\end{equation}

\section{Result and Discussion}
