\chapter{Background}
In this chapter, we provide the introduction of knowledge representation and extraction in the first section. 
In the second section, we surveyed the related work about relation extraction, including the methods of supervised learning, distant supervision, and multiple instance learning. 
As for last section, we focus on the related work of Chinese relation extraction.

\section{knowledge Representation and Extraction}
In real world, knowledge exists in different forms like text, picture, audio, video, or even in abstract form, such as someone’s memory. 
When considering textual knowledge, human could reason the meaning and retrieve the knowledge when it is needed. 
As for computers, to understand the textual knowledge, structured format is required. 
How to extract knowledge from unstructured text to structured representation remains an important issue. 
In this section, we will discuss several methods of knowledge representation and extraction.

\subsection{Knowledge Representation}

\subsection{Knowledge Extraction}
Automatic Content Extraction (ACE)\cite{ACE_intro}, as a track of Text Analysis Conference (TAC) after 2009, aims at developing novel methods to extract information from natual language text. 
In this program, entities, relation, and events are extracted.
After 2003, the program includes multilingual tracks, including English, Arabic and Chinese. 
The released data is used for supervised learning and promoted much good work. Some work related to this thesis will be discussed in subsequent section.

\subsection{Relation Extraction}

\section{Related Work of Relational Pair Extraction}

\subsection{Supervised Learning}
By using supervised learning, a set of training data is required and the extraction problem is formulated as classification problem.
When considering single relation, the problem could be viewed as binary classification problem and aims at decide whether the relation exists in given entity pairs. 

\subsubsection{Feature-Based Methods}

\subsubsection{Kernel-Based Methods}

\subsection{Distant Supervision}
To prevent ineffieiently generating training data, distant supervision is used for reducing the labelling work. 
Distant supervision uses weakly labelled training data to predict huge testing data. 
For relation extraction with distant supervision, Mintz\cite{mintz_distant} used Wikipedia as corpus and Freebase as seed pool for training a classification model. 
A strong assumption of distant supervision is that if two entities participate in one relation, then any sentence contains the two entity can represent the relation. 

\subsection{Multiple Instance Learning}
Given Distant supervision, the labelling effort is heavily reduced but cause the problem of noise when the data sentences are correlated with the seed.
For example, considering 2 sentences, ``Alice was born in Taipei'' and ``Alice went to Taipei on Saturday'' both contains the two entities ``Alice'' and ``Taipei'', but the relation in the former sentence is $WasBornIn$ while in the latter sentence is $WentTo$.
Riedel\cite{riedel_modeling} indicated that 31\% alignments of Freebase and New York Time Corpus violate the distant supervision assumption while only 13\% ones of Freebase and Wikipedia violate.

\section{Relational Pair Extraction in Chinese}

\subsection{Characteristics in Chinese Relational Pair Extraction}
\subsection{Related Work in Chinese}

